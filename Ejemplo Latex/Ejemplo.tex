% Esto es un comentario, te das cuenta porque arranca con '%'
% En Latex todo lo que arranca con '\' es algun comando para el compilador
% El compilador es el que se encargara de convertir todo el codigo en un pdf

% Indico el tipo de hoja al compilador
\documentclass[a4paper,12pt]{article}

% Paquetes que incluyo, como librerias en programacion
\usepackage[a4paper, total={190mm, 250mm}]{geometry}
\usepackage{amsmath, accents, graphicx, caption, siunitx, float}
\usepackage[bookmarks]{hyperref}

% Prototipos de comandos definidos por mi. Estos se van a usar a lo largo del documanto
\newcommand*\rfrac[2]{{}^{#1}\!/_{#2}}
\newcommand{\im}[1]{j\; #1} 
\newcommand{\parallelTwo}[2]{\left.#1\,\middle/\!\middle/\,#2\right.}


% ----------------------------------------- Arranca Finalmente el documento --------------------------------------------------
\begin{document}

% 'section{...}' le indicara al compilador que esta seccion arranca una seccion con un nombre determinado
% sabiendo esto el documento tomara una forma determinada, particularmente creara un titulo y un acceso rapido en el indice del pdf
\section{Mi titulo 1}

    % 'subsection{...}' creara un subtitulo y todo lo demas, similar al section 
    \subsection{Mi subtitulo 1}

        % Puedo escribir lo que quiera 
        Estoy escribiendo esto como muestra
        (al compilador no le importara que se encuentren en diferentes renglones aca).

        Si quisiera expresamente bajar un renglon
        \\ pondria una doble barra adelante
        \\
        \\ El compilador, por como esta hecho el doc, identara todos los renglones excepto el primero del parrafo

        Incluyo una imagen, lo meto dentro de begin{center} para que quede centrada nomas
        % El compilador se encargara de ajustar el tamano de la imagen
        \begin{center}
            \includegraphics[width=100mm]{./Logo_UTN.png}
        \end{center}

        Y puedo agregar una imagen tambien mas profecionalmente:
        \begin{figure}[H]
            \centering
		    \captionsetup{labelformat=empty}
            \includegraphics[width=100mm]{./Logo_UTN.png}
            \caption{Respuesta en Matlab (mismo orden de respuestas que grafico LTSpice)}
        \end{figure}


    % newpage indica que quiero que lo demas empiece en otra pag
    \newpage
    \subsection{Ecuaciones}

        \subsubsection{Base}

            Las ecuaciones en Latex pueden estar en el renglon $1 + 3 = 4$ \\
            O ser un componente aparte:
            \begin{gather*}
                1 +       3 = 4
            \end{gather*}
            % Dentro del gather me parece que no se admiten renglones en blanco, evitenlos

            Vemos de esta forma que dentro del texto que compone a la ecuacion, al compilador no le interesan los espacios que coloquemos.
            Para indicar un espacio aqui deberemos indicarlo por un caracter especial. Existen varios dependiendo de la longitud que queramos
            agregar, pero yo normalmente uso el \;, y si quiero mas espacio los apilo \;\;\;\;\; asi. \\
            
            Podemos poner subindices tambien: $V_1(t)$ \\
            Y si el subindice es mas largo que un caracter: $V_{subindice}$ \\

        \subsubsection{Avanzado}

            Para hacer ecuaciones mas piolas hay que empezar a meterse con funciones para el compilador. \\
            Por ejemplo podemos hacer una especie de fraccion (existen varios tipos, yo uso estas que son las mas piolas):
            \begin{gather*}
                I(t) = \cfrac{V(t)}{R} \\
                I(t) = \rfrac{V(t)}{R}
            \end{gather*}
            
            Podemos agregar una flecha como si fuera un "por lo tanto":
            \begin{gather*}
                I(t) = \cfrac{V(t)}{R}
                \Longrightarrow
                V(t) = I(t) \; \cdot \; R   % cdot me hace el punto piola que indica multiplicacion
            \end{gather*}

            Podemos colocar unidades escribiendolas o, como lo hago normalmente yo para que quede mas piola, diciendole 
            al compilador la unidad y que este s encargue. \\
            Por ejemplo: $10 Hz$ y sino $\SI{10}{\hertz}$ \\

            Ota cosa que me hice para que quede piola es el paralelo: $R_12 = \parallelTwo{R_1}{R_2}$ \\
            O tambien si tengo que colocar un numero imaginario hacer $1 + \im{2}$ en lugar de $1 + j2$. 
            Esto de hincha nomas, el comando me lo invente yo jaja. \\


            Y bueno tambien se pueden hacer matrices, raices y todos los chicches. \\
            Aca les dejo unos ejemplos:

            % Ejemplo ecuacion loca
            \begin{gather*}
                Z_T(w) = \im{X_L} + \big( \parallelTwo{R}{(-\im{X_C})} \big)
                    = \im{wL} + \bigg( \cfrac{1}{R} + jwC \bigg)^{-1} = \im{wL} + \cfrac{R}{1+jwCR} \\
                Z_T(w) = \im{wL} + \cfrac{R-\im{wCR^2}} {1+w^2C^2R^2}
                    = \cfrac{ \im{wL} + \im{w^3LC^2R^2} + R - \im{wCR^2} } { 1 + w^2C^2R^2 } \\
                Z_T(w) = \cfrac{ R + \im{( w^3LC^2R^2 + wL - wCR^2 )} } { 1 + w^2C^2R^2 }
            \end{gather*}

            % Ejemplo matriz
            \begin{gather*}
                \begin{bmatrix} \accentset{\circ}{ X_1}(t) \\ \\ \accentset{\circ}{ X_2}(t) \end{bmatrix} =
                    \begin{bmatrix} -\cfrac{1}{RC}  &\; \cfrac{1}{C} \\[10pt] -\cfrac{1}{L} &\; 0 \end{bmatrix} 
                    \begin{bmatrix} X_1(t) \\\\  X_2(t) \end{bmatrix}
                    + \begin{bmatrix} \cfrac{1}{RC} \\[10pt] \cfrac{1}{L} \end{bmatrix} \; V_i(t)
                \\[10pt]
                \begin{bmatrix} V_o(t) \\[10pt] V_L(t) \\[10pt] i_L(t) \\[10pt] i_C(t) \\[10pt] i_R(t) \end{bmatrix} = 
                    \begin{bmatrix} -1 &\; 0 \\[10pt] 1 &\; 0 \\[10pt] 3 &\; 1 \\[10pt] \rfrac{-1}{R} &\; 1 \\[10pt] \rfrac{-1}{R} &\; 0 \end{bmatrix}
                    \begin{bmatrix} X_1(t) \\\\  X_2(t) \end{bmatrix}
                    + \begin{bmatrix} 1 \\[10pt] 0 \\[10pt] 0 \\[10pt] \rfrac{1}{R} \\[10pt] \rfrac{1}{R}\; \end{bmatrix} \; V_i(t)
            \end{gather*}
\end{document}
% ------------------------------------------------- Fin del documento --------------------------------------------------